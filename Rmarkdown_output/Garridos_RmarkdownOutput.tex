% Options for packages loaded elsewhere
\PassOptionsToPackage{unicode}{hyperref}
\PassOptionsToPackage{hyphens}{url}
%
\documentclass[
]{article}
\usepackage{amsmath,amssymb}
\usepackage{lmodern}
\usepackage{iftex}
\ifPDFTeX
  \usepackage[T1]{fontenc}
  \usepackage[utf8]{inputenc}
  \usepackage{textcomp} % provide euro and other symbols
\else % if luatex or xetex
  \usepackage{unicode-math}
  \defaultfontfeatures{Scale=MatchLowercase}
  \defaultfontfeatures[\rmfamily]{Ligatures=TeX,Scale=1}
\fi
% Use upquote if available, for straight quotes in verbatim environments
\IfFileExists{upquote.sty}{\usepackage{upquote}}{}
\IfFileExists{microtype.sty}{% use microtype if available
  \usepackage[]{microtype}
  \UseMicrotypeSet[protrusion]{basicmath} % disable protrusion for tt fonts
}{}
\makeatletter
\@ifundefined{KOMAClassName}{% if non-KOMA class
  \IfFileExists{parskip.sty}{%
    \usepackage{parskip}
  }{% else
    \setlength{\parindent}{0pt}
    \setlength{\parskip}{6pt plus 2pt minus 1pt}}
}{% if KOMA class
  \KOMAoptions{parskip=half}}
\makeatother
\usepackage{xcolor}
\usepackage[margin=1in]{geometry}
\usepackage{color}
\usepackage{fancyvrb}
\newcommand{\VerbBar}{|}
\newcommand{\VERB}{\Verb[commandchars=\\\{\}]}
\DefineVerbatimEnvironment{Highlighting}{Verbatim}{commandchars=\\\{\}}
% Add ',fontsize=\small' for more characters per line
\usepackage{framed}
\definecolor{shadecolor}{RGB}{248,248,248}
\newenvironment{Shaded}{\begin{snugshade}}{\end{snugshade}}
\newcommand{\AlertTok}[1]{\textcolor[rgb]{0.94,0.16,0.16}{#1}}
\newcommand{\AnnotationTok}[1]{\textcolor[rgb]{0.56,0.35,0.01}{\textbf{\textit{#1}}}}
\newcommand{\AttributeTok}[1]{\textcolor[rgb]{0.77,0.63,0.00}{#1}}
\newcommand{\BaseNTok}[1]{\textcolor[rgb]{0.00,0.00,0.81}{#1}}
\newcommand{\BuiltInTok}[1]{#1}
\newcommand{\CharTok}[1]{\textcolor[rgb]{0.31,0.60,0.02}{#1}}
\newcommand{\CommentTok}[1]{\textcolor[rgb]{0.56,0.35,0.01}{\textit{#1}}}
\newcommand{\CommentVarTok}[1]{\textcolor[rgb]{0.56,0.35,0.01}{\textbf{\textit{#1}}}}
\newcommand{\ConstantTok}[1]{\textcolor[rgb]{0.00,0.00,0.00}{#1}}
\newcommand{\ControlFlowTok}[1]{\textcolor[rgb]{0.13,0.29,0.53}{\textbf{#1}}}
\newcommand{\DataTypeTok}[1]{\textcolor[rgb]{0.13,0.29,0.53}{#1}}
\newcommand{\DecValTok}[1]{\textcolor[rgb]{0.00,0.00,0.81}{#1}}
\newcommand{\DocumentationTok}[1]{\textcolor[rgb]{0.56,0.35,0.01}{\textbf{\textit{#1}}}}
\newcommand{\ErrorTok}[1]{\textcolor[rgb]{0.64,0.00,0.00}{\textbf{#1}}}
\newcommand{\ExtensionTok}[1]{#1}
\newcommand{\FloatTok}[1]{\textcolor[rgb]{0.00,0.00,0.81}{#1}}
\newcommand{\FunctionTok}[1]{\textcolor[rgb]{0.00,0.00,0.00}{#1}}
\newcommand{\ImportTok}[1]{#1}
\newcommand{\InformationTok}[1]{\textcolor[rgb]{0.56,0.35,0.01}{\textbf{\textit{#1}}}}
\newcommand{\KeywordTok}[1]{\textcolor[rgb]{0.13,0.29,0.53}{\textbf{#1}}}
\newcommand{\NormalTok}[1]{#1}
\newcommand{\OperatorTok}[1]{\textcolor[rgb]{0.81,0.36,0.00}{\textbf{#1}}}
\newcommand{\OtherTok}[1]{\textcolor[rgb]{0.56,0.35,0.01}{#1}}
\newcommand{\PreprocessorTok}[1]{\textcolor[rgb]{0.56,0.35,0.01}{\textit{#1}}}
\newcommand{\RegionMarkerTok}[1]{#1}
\newcommand{\SpecialCharTok}[1]{\textcolor[rgb]{0.00,0.00,0.00}{#1}}
\newcommand{\SpecialStringTok}[1]{\textcolor[rgb]{0.31,0.60,0.02}{#1}}
\newcommand{\StringTok}[1]{\textcolor[rgb]{0.31,0.60,0.02}{#1}}
\newcommand{\VariableTok}[1]{\textcolor[rgb]{0.00,0.00,0.00}{#1}}
\newcommand{\VerbatimStringTok}[1]{\textcolor[rgb]{0.31,0.60,0.02}{#1}}
\newcommand{\WarningTok}[1]{\textcolor[rgb]{0.56,0.35,0.01}{\textbf{\textit{#1}}}}
\usepackage{graphicx}
\makeatletter
\def\maxwidth{\ifdim\Gin@nat@width>\linewidth\linewidth\else\Gin@nat@width\fi}
\def\maxheight{\ifdim\Gin@nat@height>\textheight\textheight\else\Gin@nat@height\fi}
\makeatother
% Scale images if necessary, so that they will not overflow the page
% margins by default, and it is still possible to overwrite the defaults
% using explicit options in \includegraphics[width, height, ...]{}
\setkeys{Gin}{width=\maxwidth,height=\maxheight,keepaspectratio}
% Set default figure placement to htbp
\makeatletter
\def\fps@figure{htbp}
\makeatother
\setlength{\emergencystretch}{3em} % prevent overfull lines
\providecommand{\tightlist}{%
  \setlength{\itemsep}{0pt}\setlength{\parskip}{0pt}}
\setcounter{secnumdepth}{-\maxdimen} % remove section numbering
\ifLuaTeX
  \usepackage{selnolig}  % disable illegal ligatures
\fi
\IfFileExists{bookmark.sty}{\usepackage{bookmark}}{\usepackage{hyperref}}
\IfFileExists{xurl.sty}{\usepackage{xurl}}{} % add URL line breaks if available
\urlstyle{same} % disable monospaced font for URLs
\hypersetup{
  pdftitle={Garridos\_SampleOutput},
  pdfauthor={Charlene Garridos},
  hidelinks,
  pdfcreator={LaTeX via pandoc}}

\title{Garridos\_SampleOutput}
\author{Charlene Garridos}
\date{04-12-22}

\begin{document}
\maketitle

\hypertarget{problem-set-1}{%
\section{Problem Set 1}\label{problem-set-1}}

\begin{Shaded}
\begin{Highlighting}[]
\NormalTok{You go to the store and buy x bags of carrots and y bananas.  Each bag of carrots costs }\SpecialCharTok{$}\FloatTok{1.50}\NormalTok{ and each banana is }\SpecialCharTok{$}\DecValTok{0}\NormalTok{.}\FloatTok{25.}\NormalTok{  You spend }\SpecialCharTok{$}\DecValTok{6}\NormalTok{.}\FloatTok{50.}\NormalTok{  The total number of items you purchase is }\FloatTok{11.}\NormalTok{ How many bags of carrots did you buy?  How many bananas did you buy? }
\end{Highlighting}
\end{Shaded}

\begin{enumerate}
\def\labelenumi{\arabic{enumi}.}
\tightlist
\item
  Write an equation with x and y.
\item
  Find the value of the following:

  \begin{enumerate}
  \def\labelenumii{\alph{enumii}.}
  \tightlist
  \item
    x
  \item
    y
  \end{enumerate}
\end{enumerate}

\hypertarget{solution}{%
\subsection{Solution}\label{solution}}

Given the information, we have 2 equations. We know each bag of carrots
is \$1.50 and each banana is \$0.25. We also know the total amount we
spend is \$6.50. So, we can write the equation

1.5x+0.25y=6.5

where x is the number of bags of carrots and y is the number of bananas.

We also know the total number of items we purchased is 11. We can write
the equation as

x+y=11

where x is number of bags of carrots and y is the number of bananas.

To solve, we will solve for one variable in one equation and substitute
it into the other equation. So,

x+y=11

y=11−x

Now, we can substitute the value of y into the first equation. We get,

1.5x+0.25(11−x)=6.5

We distribute.

1.5x+2.75−0.25x=6.5

We combine like terms.

1.25x+2.75=6.5

We solve for x by getting x alone.

1.25x+2.75−2.75=6.5−2.75

1.25x=3.75

x=3

Therefore, the number of bags of carrots we bought is 3. To find the
number of bananas, we simply substitute x into the equation.

x+y=11

3+y=11

3+y−3=11−3

y=11−3

y=8

Therefore, the number of bananas we bought is 8.

So we bought 3 bags of carrots and 8 bananas.

Reference:
\url{https://www.varsitytutors.com/prealgebra-help/word-problems-with-two-unknowns}

\hypertarget{data-visualization}{%
\section{Data Visualization}\label{data-visualization}}

\begin{Shaded}
\begin{Highlighting}[]
\FunctionTok{plot}\NormalTok{(}\FunctionTok{rnorm}\NormalTok{(}\DecValTok{150}\NormalTok{))}
\end{Highlighting}
\end{Shaded}

\includegraphics{Garridos_SampleOutput_files/figure-latex/unnamed-chunk-2-1.pdf}

\hypertarget{data-set}{%
\subsection{Data Set}\label{data-set}}

\ldots\ldots..

\hypertarget{acknowledgement}{%
\section{Acknowledgement}\label{acknowledgement}}

I'd want to express gratitude to the speaker, Sir Cuenca, Sir Benitez as
well as the facilitators and those who helped make this workshop
possible. I sincerely appreciate your time, effort, and passion in
educating us. Thank you so much for an excellent and amusing lecture po.

\end{document}
